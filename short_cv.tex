%%%%%%%%%%%%%%%%%%%%%%%%%%%%%%%%%%%%%%%%%%%%%%%%%%%%%%%%%%%%%%%%%%%%%%%%
%%%%%%%%%%%%%%%%%%%%%% Simple LaTeX CV Template %%%%%%%%%%%%%%%%%%%%%%%%
%%%%%%%%%%%%%%%%%%%%%%%%%%%%%%%%%%%%%%%%%%%%%%%%%%%%%%%%%%%%%%%%%%%%%%%%

%%%%%%%%%%%%%%%%%%%%%%%%%%%%%%%%%%%%%%%%%%%%%%%%%%%%%%%%%%%%%%%%%%%%%%%%
%% NOTE: If you find that it says                                     %%
%%                                                                    %%
%%                           1 of ??                                  %%
%%                                                                    %%
%% at the bottom of your first page, this means that the AUX file     %%
%% was not available when you ran LaTeX on this source. Simply RERUN  %% 
%% LaTeX to get the ``??'' replaced with the number of the last page  %% 
%% of the document. The AUX file will be generated on the first run   %%
%% of LaTeX and used on the second run to fill in all of the          %%
%% references.                                                        %%
%%%%%%%%%%%%%%%%%%%%%%%%%%%%%%%%%%%%%%%%%%%%%%%%%%%%%%%%%%%%%%%%%%%%%%%%

%%%%%%%%%%%%%%%%%%%%%%%%%%%% Document Setup %%%%%%%%%%%%%%%%%%%%%%%%%%%%

% Don't like 10pt? Try 11pt or 12pt
\documentclass[11pt]{article}

% This is a helpful package that puts math inside length specifications
\usepackage{calc}
%\usepackage[latin1]{inputenc}
\usepackage[T1]{fontenc}

% Use Different Font 
%\usepackage[scaled]{helvet}
\usepackage{libertine}
%\usepackage{palatino}
\renewcommand*\familydefault{\sfdefault} %% Only if the base font of the document is to be sans serif
\usepackage[T1]{fontenc}


\usepackage{revnum}

% Layout: Puts the section titles on left side of page
\reversemarginpar

%
%         PAPER SIZE, PAGE NUMBER, AND DOCUMENT LAYOUT NOTES:
%
% The next \usepackage line changes the layout for CV style section
% headings as marginal notes. It also sets up the paper size as either
% letter or A4. By default, letter was used. If A4 paper is desired,
% comment out the letterpaper lines and uncomment the a4paper lines.
%
% As you can see, the margin widths and section title widths can be
% easily adjusted.
%
% ALSO: Notice that the includefoot option can be commented OUT in order
% to put the PAGE NUMBER *IN* the bottom margin. This will make the
% effective text area larger.
%
% IF YOU WISH TO REMOVE THE ``of LASTPAGE'' next to each page number,
% see the note about the +LP and -LP lines below. Comment out the +LP
% and uncomment the -LP.
%
% IF YOU WISH TO REMOVE PAGE NUMBERS, be sure that the includefoot line
% is uncommented and ALSO uncomment the \pagestyle{empty} a few lines
% below.
%

%% Use these lines for letter-sized paper
\usepackage[paper=letterpaper,
            %includefoot, % Uncomment to put page number above margin
            marginparwidth=1.1in,     % Length of section titles
            marginparsep=0.005in,       % Space between titles and text
            margin=1in,               % 1 inch margins
            includemp]{geometry}

%% Use these lines for A4-sized paper
%\usepackage[paper=a4paper,
%            %includefoot, % Uncomment to put page number above margin
%            marginparwidth=30.5mm,    % Length of section titles
%            marginparsep=1.5mm,       % Space between titles and text
%            margin=25mm,              % 25mm margins
%            includemp]{geometry}

%% More layout: Get rid of indenting throughout entire document
\setlength{\parindent}{0in}

%% This gives us fun enumeration environments. compactitem will be nice.
\usepackage{paralist}

%% Reference the last page in the page number
%
% NOTE: comment the +LP line and uncomment the -LP line to have page
%       numbers without the ``of ##'' last page reference)
%
% NOTE: uncomment the \pagestyle{empty} line to get rid of all page
%       numbers (make sure includefoot is commented out above)
%
\usepackage{amsmath,amssymb,amsfonts,textcomp}

\usepackage{fancyhdr,lastpage}
\pagestyle{fancy}
%\pagestyle{empty}      % Uncomment this to get rid of page numbers
\fancyhf{}\renewcommand{\headrulewidth}{0pt}
\fancyfootoffset{\marginparsep+\marginparwidth}
\newlength{\footpageshift}
\setlength{\footpageshift}
          {0.5\textwidth+0.5\marginparsep+0.5\marginparwidth-2in}
\lfoot{\hspace{\footpageshift}%
       \parbox{4in}{\, \hfill %
                    \arabic{page} of \protect\pageref*{LastPage} % +LP
%                    \arabic{page}                               % -LP
                    \hfill \,}}

% Finally, give us PDF bookmarks
\usepackage{color,hyperref}
\definecolor{darkblue}{rgb}{0.0,0.0,0.3}
\hypersetup{colorlinks,breaklinks,
            linkcolor=darkblue,urlcolor=darkblue,
            anchorcolor=darkblue,citecolor=darkblue}

%	Kuler
%	copy of pomegranate explosion
\definecolor{blue}{RGB}{99,166,159}
\definecolor{yellow}{RGB}{242,225,172}
\definecolor{orange}{RGB}{242,131,107}
\definecolor{pink}{RGB}{242,89,75}
\definecolor{red}{RGB}{205,44,36}


%%%%%%%%%%%%%%%%%%%%%%%% End Document Setup %%%%%%%%%%%%%%%%%%%%%%%%%%%%


%%%%%%%%%%%%%%%%%%%%%%%%%%% Helper Commands %%%%%%%%%%%%%%%%%%%%%%%%%%%%

% The title (name) with a horizontal rule under it
%
% Usage: \makeheading{name}
%
% Place at top of document. It should be the first thing.
\newcommand{\makeheading}[1]%
        {\hspace*{-\marginparsep minus \marginparwidth}%
         \begin{minipage}[t]{\textwidth+\marginparwidth+\marginparsep}%
                {\large \bfseries #1}\\[-0.15\baselineskip]%
%                {\LARGE \bfseries #1}\\[-0.15\baselineskip]%
                 \rule{\columnwidth}{1pt}%
         \end{minipage}}

% The section headings
%
% Usage: \section{section name}
%
% Follow this section IMMEDIATELY with the first line of the section
% text. Do not put whitespace in between. That is, do this:
%
%       \section{My Information}
%       Here is my information.
%
% and NOT this:
%
%       \section{My Information}
%
%       Here is my information.
%
% Otherwise the top of the section header will not line up with the top
% of the section. Of course, using a single comment character (%) on
% empty lines allows for the function of the first example with the
% readability of the second example.
\renewcommand{\section}[2]%
        {\pagebreak[2]\vspace{1.3\baselineskip}%
         \phantomsection\addcontentsline{toc}{section}{#1}%
         \hspace{0in}%
         \marginpar{%
 \raggedright\scshape#1}#2}

% An itemize-style list with lots of space between items
\newenvironment{outerlist}[1][\enskip\textbullet]%
        {\begin{itemize}[#1]}{\end{itemize}%
         \vspace{-.6\baselineskip}}

% An environment IDENTICAL to outerlist that has better pre-list spacing
% when used as the first thing in a \section 
\newenvironment{lonelist}[1][\enskip\textbullet]%
        {\vspace{-\baselineskip}\begin{list}{#1}{%
        \setlength{\partopsep}{0pt}%
        \setlength{\topsep}{0pt}}}
        {\end{list}\vspace{-.6\baselineskip}}

% An itemize-style list with little space between items
\newenvironment{innerlist}[1][\enskip\textbullet]%
        {\begin{compactitem}[#1]}{\end{compactitem}}

% To add some paragraph space between lines.
% This also tells LaTeX to preferably break a page on one of these gaps
% if there is a needed pagebreak nearby.
\newcommand{\blankline}{\quad\pagebreak[2]}


\def\Vhrulefill{\leavevmode\leaders\hrule height 0.7ex depth \dimexpr0.4pt-0.7ex\hfill\kern0pt}

%%%%%%%%%%%%%%%%%%%%%%%% End Helper Commands %%%%%%%%%%%%%%%%%%%%%%%%%%%

%%%%%%%%%%%%%%%%%%%%%%%%% Begin CV Document %%%%%%%%%%%%%%%%%%%%%%%%%%%%

\begin{document}
%\sffamily

\color{red}
\makeheading{Biographical Sketch: Nicholas J. Mayhall}
\color{black}

\section{\textbf{\color{red}Contact Information}}
%
% NOTE: Mind where the & separators and \\ breaks are in the following
%       table.
%
% ALSO: \rcollength is the width of the right column of the table 
%       (adjust it to your liking; default is 1.85in).
%
\newlength{\rcollength}\setlength{\rcollength}{2.85in}%
%
\begin{tabular}[t]{@{}p{\textwidth-\rcollength}p{\rcollength}}
Department of Chemistry (0212)         	& Voice: (540) 231-3298 \\
480 Davidson Hall                 	& E-mail: \href{mailto:nmayhall@vt.edu}{nmayhall@vt.edu}\\
Virginia Tech      \\   
900 W. Campus Drive \\
Blacksburg, VA 24061 & \\

	
	 
\end{tabular}



%\section{Citizenship}
%
%USA
%

%\section{\textbf{Education}}
\section{\textbf{\color{red}Education and Training}}
%
\begin{tabular}{p{40pt} p{150pt} p{110pt} p{60pt}}
Postdoc	& University of CA, Berkeley  	&	&	2011-2015 
\end{tabular}\newline
\begin{tabular}{p{40pt} p{150pt} p{110pt} p{60pt}}
Ph.D.	& Indiana University  	&	Computational Chemistry	&	2011 
\end{tabular}\newline
\begin{tabular}{p{40pt} p{150pt} p{110pt} p{60pt}}
B.S.	& University of Southern Indiana  	&	Chemistry	&	2006
\end{tabular}


\section{\textbf{\color{red}Experience}}
%\begin{tabular}{p{40pt} p{100pt} p{100pt} p{60pt}}
\begin{tabular}{p{40pt} p{150pt} p{110pt} p{60pt}}
Assistant Professor	& Virginia Tech 	&	Chemistry	& 2015-Present	
\end{tabular}


\blankline


%\newpage
%\blankline

\section{\textbf{\color{red}Selected Relevant Publications}}
\begin{enumerate}
\item \textbf{From model Hamiltonians to ab initio Hamiltonians and back again: Using single excitation quantum chemistry methods to find multiexciton states in singlet fission materials}\\
N. J. Mayhall\\
\textsl{Journal of Chemical Theory and Computation}, 12, 4263-4273, (2016)

\item \textbf{Computational Quantum Chemistry For Multiple Site Heisenberg Spin Couplings Made Simple: Still Only One Spin Flip Required}\\
N. J. Mayhall and M. Head-Gordon\\
\textsl{Journal of Physical Chemistry Letters}, 6, 1982-1988, (2015)

\item \textbf{Increasing spin-flips and decreasing cost: Perturbative corrections for external singles to the complete 
		active space spin flip model for low-lying excited states and strong correlation}\\
N. J. Mayhall and M. Head-Gordon\\
\textsl{The Journal of Chemical Physics}, 141, 044112, (2014)

\item \textbf{Spin-Flip Non-Orthogonal Configuration Interaction: A variational and almost black-box method for describing strong correlation}\\
N. J. Mayhall, P. Horn, E. J. Sundstrom, and M. Head-Gordon\\
\textsl{Physical Chemistry Chemical Physics}, 16, 22694-22705, (2014)


\item \textbf{A Quasidegenerate Second-Order Perturbation Theory Approximation to RAS-\textsl{n}SF for Excited States and Strong Correlations}\\
N. J. Mayhall, M. Goldey, and M. Head-Gordon\\
\textsl{Journal of Chemical Theory and Computation}, 10, 589-599, (2014)


\item \textbf{Computational quantum chemistry for single Heisenberg spin couplings made simple: Just one spin flip required}\\
N. J. Mayhall and M. Head-Gordon\\
\textsl{Journal of Chemical Physics}, 141, 134111, (2014)


\item 
\textbf{Many-Overlapping-Body (MOB) Expansion: A Generalized Many Body Expansion for Nondisjoint Monomers in Molecular Fragmentation Calculations of Covalent Molecules} \\
N. J. Mayhall and K. Raghavachari \\
\textsl{Journal of Chemical Theory and Computation}, 8, 2669-2675, (2012)

\item \textbf{Molecules-in-Molecules: A Hybrid-Energy Fragmentation Approach for Accurate Calculations on Large Molecules and Materials}\\
N. J. Mayhall and K. Raghavachari\\
\textsl{Journal of Chemical Theory and Computation}, 7, 1336-1343, (2011)

\item \textbf{Charge transfer across ONIOM QM:QM boundaries: The impact of model system preparation}\\
N. J. Mayhall and K. Raghavachari\\
 \textsl{Journal of Chemical Theory and Computation}, 6, 3131-3136 (2010)

\item \textbf{ONIOM-based QM:QM electronic embedding method using L\"owdin atomic charges: Energies and analytic gradients}\\
N. J. Mayhall, K. Raghavachari, H. P. Hratchian\\
\textsl{Journal of Chemical Physics}, 132, 114107 (2010)

%\end{revnumerate}
\end{enumerate}

\blankline

\section{\textbf{\color{red}Synergistic Activities}}
\begin{lonelist}
	  \setlength\itemsep{1pt}
	  \item[] Supervising 3 PhD students (Houck,Abraham,Shaikh) 
	  \item[] Taught undergraduate Physical Chemistry second semester (quantum) CHEM 3616 (01/17-05/17), and graduate Quantum Chemistry CHEM 6634 (08/15-12/15 and 08/16-08/16).
	  \item[] 5 invited talks since 2015
\item[] Journal of Chemical Physics, Journal of Chemical Theory and Computation,  Journal of Physical Chemistry, Journal of Physical Chemistry Letters, Molecular Physics, Chemical Physics Letters 
\end{lonelist}

\blankline

\section{\textbf{\color{red}Collaborators and other affiliations}}
\begin{lonelist}
	  \setlength\itemsep{1pt}
	  \item[] Martin Head-Gordon (UC Berkeley), Amanda Morris (Virginia Tech)
		  P. Horn (Google),  E. J. Sundstrom (Zenefits), M. Goldey (Univ. of Chicago),
\end{lonelist}
\blankline
\blankline
\blankline

\section{\textbf{\color{red}Graduate and Postdoctoral Advisors and Advisees}}
\begin{lonelist}
	  \setlength\itemsep{1pt}
	  \item[] PhD Advisor: Krishnan Raghavachari (Indiana University)
	  \item[] Postdoc Advisor: Martin Head-Gordon (UC Berkeley)
	  \item[] Graduate student Advisees: Vibin Abraham (current),  Shannon Houck (current)
\end{lonelist}

\blankline
%\pagebreak




\end{document}

%%%%%%%%%%%%%%%%%%%%%%%%%% End CV Document %%%%%%%%%%%%%%%%%%%%%%%%%%%%%
