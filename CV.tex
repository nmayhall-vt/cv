%%%%%%%%%%%%%%%%%%%%%%%%%%%%%%%%%%%%%%%%%%%%%%%%%%%%%%%%%%%%%%%%%%%%%%%%
%%%%%%%%%%%%%%%%%%%%%% Simple LaTeX CV Template %%%%%%%%%%%%%%%%%%%%%%%%
%%%%%%%%%%%%%%%%%%%%%%%%%%%%%%%%%%%%%%%%%%%%%%%%%%%%%%%%%%%%%%%%%%%%%%%%

%%%%%%%%%%%%%%%%%%%%%%%%%%%%%%%%%%%%%%%%%%%%%%%%%%%%%%%%%%%%%%%%%%%%%%%%
%% NOTE: If you find that it says                                     %%
%%                                                                    %%
%%                           1 of ??                                  %%
%%                                                                    %%
%% at the bottom of your first page, this means that the AUX file     %%
%% was not available when you ran LaTeX on this source. Simply RERUN  %% 
%% LaTeX to get the ``??'' replaced with the number of the last page  %% 
%% of the document. The AUX file will be generated on the first run   %%
%% of LaTeX and used on the second run to fill in all of the          %%
%% references.                                                        %%
%%%%%%%%%%%%%%%%%%%%%%%%%%%%%%%%%%%%%%%%%%%%%%%%%%%%%%%%%%%%%%%%%%%%%%%%

%%%%%%%%%%%%%%%%%%%%%%%%%%%% Document Setup %%%%%%%%%%%%%%%%%%%%%%%%%%%%

% Don't like 10pt? Try 11pt or 12pt
\documentclass[10pt]{article}

% This is a helpful package that puts math inside length specifications
\usepackage{calc}
%\usepackage[latin1]{inputenc}
\usepackage[T1]{fontenc}

% Use Different Font 
%\usepackage[scaled]{helvet}
\usepackage{libertine}
%\usepackage{palatino}
\renewcommand*\familydefault{\sfdefault} %% Only if the base font of the document is to be sans serif
\usepackage[T1]{fontenc}


\usepackage{revnum}

% Layout: Puts the section titles on left side of page
\reversemarginpar

%
%         PAPER SIZE, PAGE NUMBER, AND DOCUMENT LAYOUT NOTES:
%
% The next \usepackage line changes the layout for CV style section
% headings as marginal notes. It also sets up the paper size as either
% letter or A4. By default, letter was used. If A4 paper is desired,
% comment out the letterpaper lines and uncomment the a4paper lines.
%
% As you can see, the margin widths and section title widths can be
% easily adjusted.
%
% ALSO: Notice that the includefoot option can be commented OUT in order
% to put the PAGE NUMBER *IN* the bottom margin. This will make the
% effective text area larger.
%
% IF YOU WISH TO REMOVE THE ``of LASTPAGE'' next to each page number,
% see the note about the +LP and -LP lines below. Comment out the +LP
% and uncomment the -LP.
%
% IF YOU WISH TO REMOVE PAGE NUMBERS, be sure that the includefoot line
% is uncommented and ALSO uncomment the \pagestyle{empty} a few lines
% below.
%

%% Use these lines for letter-sized paper
\usepackage[paper=letterpaper,
            %includefoot, % Uncomment to put page number above margin
            marginparwidth=1.1in,     % Length of section titles
            marginparsep=0.005in,       % Space between titles and text
            margin=1in,               % 1 inch margins
            includemp]{geometry}

%% Use these lines for A4-sized paper
%\usepackage[paper=a4paper,
%            %includefoot, % Uncomment to put page number above margin
%            marginparwidth=30.5mm,    % Length of section titles
%            marginparsep=1.5mm,       % Space between titles and text
%            margin=25mm,              % 25mm margins
%            includemp]{geometry}

%% More layout: Get rid of indenting throughout entire document
\setlength{\parindent}{0in}

%% This gives us fun enumeration environments. compactitem will be nice.
\usepackage{paralist}

%% Reference the last page in the page number
%
% NOTE: comment the +LP line and uncomment the -LP line to have page
%       numbers without the ``of ##'' last page reference)
%
% NOTE: uncomment the \pagestyle{empty} line to get rid of all page
%       numbers (make sure includefoot is commented out above)
%
\usepackage{amsmath,amssymb,amsfonts,textcomp}

\usepackage{fancyhdr,lastpage}
\pagestyle{fancy}
%\pagestyle{empty}      % Uncomment this to get rid of page numbers
\fancyhf{}\renewcommand{\headrulewidth}{0pt}
\fancyfootoffset{\marginparsep+\marginparwidth}
\newlength{\footpageshift}
\setlength{\footpageshift}
          {0.5\textwidth+0.5\marginparsep+0.5\marginparwidth-2in}
\lfoot{\hspace{\footpageshift}%
       \parbox{4in}{\, \hfill %
                    \arabic{page} of \protect\pageref*{LastPage} % +LP
%                    \arabic{page}                               % -LP
                    \hfill \,}}

% Finally, give us PDF bookmarks
\usepackage{color,hyperref}
\definecolor{darkblue}{rgb}{0.0,0.0,0.3}
\hypersetup{colorlinks,breaklinks,
            linkcolor=darkblue,urlcolor=darkblue,
            anchorcolor=darkblue,citecolor=darkblue}

%	Kuler
%	copy of pomegranate explosion
\definecolor{blue}{RGB}{99,166,159}
\definecolor{yellow}{RGB}{242,225,172}
\definecolor{orange}{RGB}{242,131,107}
\definecolor{pink}{RGB}{242,89,75}
\definecolor{red}{RGB}{205,44,36}


%%%%%%%%%%%%%%%%%%%%%%%% End Document Setup %%%%%%%%%%%%%%%%%%%%%%%%%%%%


%%%%%%%%%%%%%%%%%%%%%%%%%%% Helper Commands %%%%%%%%%%%%%%%%%%%%%%%%%%%%

% The title (name) with a horizontal rule under it
%
% Usage: \makeheading{name}
%
% Place at top of document. It should be the first thing.
\newcommand{\makeheading}[1]%
        {\hspace*{-\marginparsep minus \marginparwidth}%
         \begin{minipage}[t]{\textwidth+\marginparwidth+\marginparsep}%
                {\large \bfseries #1}\\[-0.15\baselineskip]%
%                {\LARGE \bfseries #1}\\[-0.15\baselineskip]%
                 \rule{\columnwidth}{1pt}%
         \end{minipage}}

% The section headings
%
% Usage: \section{section name}
%
% Follow this section IMMEDIATELY with the first line of the section
% text. Do not put whitespace in between. That is, do this:
%
%       \section{My Information}
%       Here is my information.
%
% and NOT this:
%
%       \section{My Information}
%
%       Here is my information.
%
% Otherwise the top of the section header will not line up with the top
% of the section. Of course, using a single comment character (%) on
% empty lines allows for the function of the first example with the
% readability of the second example.
\renewcommand{\section}[2]%
        {\pagebreak[2]\vspace{1.3\baselineskip}%
         \phantomsection\addcontentsline{toc}{section}{#1}%
         \hspace{0in}%
         \marginpar{%
 \raggedright\scshape#1}#2}

% An itemize-style list with lots of space between items
\newenvironment{outerlist}[1][\enskip\textbullet]%
        {\begin{itemize}[#1]}{\end{itemize}%
         \vspace{-.6\baselineskip}}

% An environment IDENTICAL to outerlist that has better pre-list spacing
% when used as the first thing in a \section 
\newenvironment{lonelist}[1][\enskip\textbullet]%
        {\vspace{-\baselineskip}\begin{list}{#1}{%
        \setlength{\partopsep}{0pt}%
        \setlength{\topsep}{0pt}}}
        {\end{list}\vspace{-.6\baselineskip}}

% An itemize-style list with little space between items
\newenvironment{innerlist}[1][\enskip\textbullet]%
        {\begin{compactitem}[#1]}{\end{compactitem}}

% To add some paragraph space between lines.
% This also tells LaTeX to preferably break a page on one of these gaps
% if there is a needed pagebreak nearby.
\newcommand{\blankline}{\quad\pagebreak[2]}


\def\Vhrulefill{\leavevmode\leaders\hrule height 0.7ex depth \dimexpr0.4pt-0.7ex\hfill\kern0pt}

%%%%%%%%%%%%%%%%%%%%%%%% End Helper Commands %%%%%%%%%%%%%%%%%%%%%%%%%%%

%%%%%%%%%%%%%%%%%%%%%%%%% Begin CV Document %%%%%%%%%%%%%%%%%%%%%%%%%%%%

\begin{document}
%\sffamily

\color{red}
\makeheading{\Large Nicholas J. Mayhall}
\color{black}

\section{\textbf{\color{red}Contact Information}}
%
% NOTE: Mind where the & separators and \\ breaks are in the following
%       table.
%
% ALSO: \rcollength is the width of the right column of the table 
%       (adjust it to your liking; default is 1.85in).
%
\newlength{\rcollength}\setlength{\rcollength}{2.85in}%
%
\begin{tabular}[t]{@{}p{\textwidth-\rcollength}p{\rcollength}}
Department of Chemistry (0212)         	& Voice: (540) 231-3298 \\
480 Davidson Hall                 	& E-mail: \href{mailto:nmayhall@vt.edu}{nmayhall@vt.edu}\\
Virginia Tech      \\   
900 W. Campus Drive \\
Blacksburg, VA 24061 & \\

	
	 
\end{tabular}



%\section{Citizenship}
%
%USA
%

%\section{\textbf{Education}}
\section{\textbf{\color{red}Education}}
%
\begin{tabular}{p{20pt} p{190pt} p{110pt} p{20pt}}
Ph.D.	& Indiana University,  Bloomington, IN  	&	Computational Chemistry	&	2011 
\end{tabular}\newline
\begin{tabular}{p{20pt} p{190pt} p{110pt} p{20pt}}
B.S.	& University of Southern Indiana,  Evansville, IN  	&	Chemistry	&	2006
\end{tabular}


\section{\textbf{\color{red}Experience}}
\begin{tabular}{p{240pt} l}
Assistant Professor  & 2015 - Present
\end{tabular}
\begin{innerlist}
\item[\hspace*{20pt}] Virginia Tech \\
\end{innerlist}

\vspace{-8pt}
\begin{tabular}{p{240pt} l}
Post-Doctoral Associate  & 2011 - 2015 
\end{tabular}
\begin{innerlist}
\item[\hspace*{20pt}] University of CA -- Berkeley
\item[\hspace*{20pt}] Research group of Prof. Martin Head-Gordon\\
\end{innerlist}

\vspace{-8pt}

\begin{tabular}{p{240pt} l}
Graduate Research Assistant  & 2007 - 2011
\end{tabular}
\begin{innerlist}
\item[\hspace*{20pt}] Indiana University
\item[\hspace*{20pt}] Research group of Prof. Krishnan Raghavachari\\
\end{innerlist}

\vspace{-8pt}

\begin{tabular}{p{240pt} l}
Graduate Student Instructor  & 2006 - 2007
\end{tabular}
\begin{innerlist}
\item[\hspace*{20pt}] Indiana University
\item[\hspace*{20pt}] General Chemistry I Lab with Prof. Todd Stone
\item[\hspace*{20pt}] General Chemistry II with Prof. Srinivasan Iyengar\\
\end{innerlist}
%
\vspace{-8pt}

\begin{tabular}{p{240pt} l}
Undergraduate Teaching Assistant  & 2005 - 2006
\end{tabular}
\begin{innerlist}
\item[\hspace*{20pt}] University of Southern Indiana
\item[\hspace*{20pt}] Physical Chemistry I and II with Prof. Evan Millam\\
\end{innerlist}
%
\vspace{-8pt}

\begin{tabular}{p{240pt} l}
REU Undergraduate Researcher  & 2005 - 2005
\end{tabular}
\begin{innerlist}
\item[\hspace*{20pt}] University of Memphis
\item[\hspace*{20pt}] Research group of Prof. Ted Burkey\\
\end{innerlist}

%\begin{outerlist}
%\item[] Physical Chemistry,  June 2011
%	\begin{innerlist}
%	\item Advisor: Professor Krishnan Raghavachari
%	\item Thesis Topic: Hybrid Energy Methods in Computational Chemistry
%    \item Areas of Study: Physical Chemistry, Inorganic Chemistry
%%	\item Collaborators: 
%%		\begin{enumerate}
%%		\item[] Professor Caroline Chick Jarrold (Indiana University) 
%%		\item[] Larry A. Curtiss (Argonne National Laboratory)	
%%		\end{enumerate}
%%	\item Area of Study: Theoretical and Computational Chemistry
%%	\item Current GPA: 3.788
%	\end{innerlist}
%\end{outerlist}

\blankline

%B.A. 	\textsl{University of Southern Indiana}, Evansville, IN USA
%\begin{outerlist}
%\item[] B.A., Chemistry, May 2006 
%        \begin{innerlist}
%%        \item Advisor: Professor Evan Millam
%        \item Areas of Study: Chemistry, Mathematics
%%		\item Undergraduate Research: Computational Chemistry
%%        \item \textit{Cum Laude}, GPA 3.747
%        \end{innerlist}
%\end{outerlist}

%\section{\textbf{Research}}
%%
%\textsl{Developments in Hybrid Electronic Structure Methods}
%\begin{innerlist}
% \item[]  The ability of a hybrid electronic structure method to provide accurate and chemically meaningful results is often determined by the means in which the high level and low level wave functions interact. We are currently working on developing new hybrid models which capture more of the essential interactions such as electronic polarization and charge transfer while maintaining computational efficiency.
%\end{innerlist}
%
%\blankline
%
%\textsl{Small Molecule Activation by Transition Metal Oxides}
%\begin{innerlist}
%\item[] We study both structural properties and reactions of transition metal oxides. In a collaborative effort with Professor Caroline Jarrold's experimental research group, we probe the reactive interactions between small molecules such as methane and water with transition metal oxide clusters.  
%\end{innerlist}
%
%\blankline
%
%\textsl{Accurate Calculation of Transition Metal Thermochemistry}
%\begin{innerlist}
%\item[] Working with Larry Curtiss at Argonne National Laboratory, we are developing new composite model chemistries which are capable of computing accurate thermochemical properties of transition metal complexes. 
%\end{innerlist}


\section{\textbf{\color{red}Awards}}
%
\begin{tabular}{p{240pt} p{80pt} l}
NSF Career Award & VT & 2018
\end{tabular}
\newline
\begin{tabular}{p{240pt} p{80pt} l}
ACS PHYS Division Postdoctoral Research Awards & UC Berkeley & 2014
\end{tabular}
\newline
\begin{tabular}{p{240pt} p{80pt} l}
Richard Slagle Fellowship  &  Indiana University & 2010
\end{tabular}
\newline
\begin{tabular}{p{240pt} p{80pt} l}
 E.M. Kratz Fellowship  &  Indiana University & 2009
\end{tabular}
\newline
\begin{tabular}{p{240pt} p{80pt} l}
 Felix Haurowitz Award  &  Indiana University & 2009
\end{tabular}
\newline
\begin{tabular}{p{240pt} p{80pt} l}
E. Campaigne C500 Award  &  Indiana University & 2008
\end{tabular}
\newline
\begin{tabular}{p{240pt} p{80pt} l}
Academic Achievement Award for Chemistry  &  USI & 2006
\end{tabular}
\newline
\begin{tabular}{p{240pt} p{80pt} l}
Outstanding Achievement Award in Physical Chemistry  &  USI & 2005
\end{tabular}
\newline
\begin{tabular}{p{240pt} p{80pt} l}
Integra Bank Distinguished Professor Scholar  &  USI & 2005
\end{tabular}
\newline
\begin{tabular}{p{240pt} p{80pt} l}
O. John Logsdon Chemistry Scholarship  &  USI & 2005
\end{tabular}
\newline
\begin{tabular}{p{240pt} p{80pt} l}
CRC Freshman Chemistry Achievement Award  &  USI & 2003
\end{tabular}

%\newpage
\blankline

\section{\textbf{\color{red}Publications}}
\begin{lonelist}


%\item[26\hspace{4pt}] \textbf{A simple way to accurately compute multiexciton states in singlet fission materials}\\
%N. J. Mayhall\\
%\textsl{TBD}, in prep (2016)
%
%\item[25\hspace{4pt}] \textbf{Adding dynamical correlation to spin-complete spin-flip wavefunctions via second-order perturbation theory}\\
%N. J. Mayhall\\
%\textsl{TBD}, in prep (2016)

\item[27\hspace{4pt}] \textbf{A Simple Rule to Predict Boundedness of Multi-Exciton States in Covalently-Linked Singlet Fission Dimers}\\
V. Abraham and N. J. Mayhall\\
\textsl{Journal of Physical Chemistry Letters}, 8, 5472-5478, (2017)

\newpage

\item[26\hspace{4pt}] \textbf{Using Higher-Order Singular Value Decomposition To Define Weakly Coupled and Strongly Correlated Clusters: The $n$-Body Tucker Approximation}\\
N. J. Mayhall\\
\textsl{Journal of Chemical Theory and Computation}, 13, 4818-4828, (2017)

\item[25\hspace{4pt}] \textbf{From model Hamiltonians to ab initio Hamiltonians and back again: Using single excitation quantum chemistry methods to find multiexciton states in singlet fission materials}\\
N. J. Mayhall\\
\textsl{Journal of Chemical Theory and Computation}, 12, 4263-4273, (2016)

\item[24\hspace{4pt}] \textbf{Computational Quantum Chemistry For Multiple Site Heisenberg Spin Couplings Made Simple: Still Only One Spin Flip Required}\\
N. J. Mayhall and M. Head-Gordon\\
\textsl{Journal of Physical Chemistry Letters}, 6, 1982-1988, (2015)


\item[23\hspace{4pt}] \textbf{Advances in molecular quantum chemistry contained in the Q-Chem 4 program package}\\
Y. Shao, et al.\\
\textsl{Molecular Physics}, 113, 184-215, (2014)

\item[22\hspace{4pt}] \textbf{Computational quantum chemistry for single Heisenberg spin couplings made simple: Just one spin flip required}\\
N. J. Mayhall and M. Head-Gordon\\
\textsl{Journal of Chemical Physics}, 141, 134111, (2014)



%\begin{center}
%	\Vhrulefill{} \textbf{In Print} {} \Vhrulefill
%\end{center}
%
\item[21\hspace{4pt}] \textbf{Spin-Flip Non-Orthogonal Configuration Interaction: A variational and almost black-box method for describing strong correlation}\\
N. J. Mayhall, P. Horn, E. J. Sundstrom, and M. Head-Gordon\\
\textsl{Physical Chemistry Chemical Physics}, 16, 22694-22705, (2014)


\item[20\hspace{4pt}] \textbf{Increasing spin-flips and decreasing cost: Perturbative corrections for external singles to the complete 
		active space spin flip model for low-lying excited states and strong correlation}\\
N. J. Mayhall and M. Head-Gordon\\
\textsl{The Journal of Chemical Physics}, 141, 044112, (2014)

\item[19\hspace{4pt}] \textbf{A Quasidegenerate Second-Order Perturbation Theory Approximation to RAS-\textsl{n}SF for Excited States and Strong Correlations}\\
N. J. Mayhall, M. Goldey, and M. Head-Gordon\\
\textsl{Journal of Chemical Theory and Computation}, 10, 589-599, (2014)

\item[18\hspace{4pt}] \textbf{On the Formation of Silacyclopropenylidene ($c$-SiC$_2$H$_2$) and its Role in the Organosilicon Chemistry in the Interstellar Medium}\\
D. S. N. Parker, A. V. Wilson, R. I. Kaiser, N. J. Mayhall, M. Head-Gordon, and A. G. G. M. Tielens\\
\textsl{The Astrophysical Journal}, 770, 33, (2013)

\item[17\hspace{4pt}] \textbf{A Composite Energy Treatment for Sterically Hindered Cluster Models for the Si(100) Surface}\\
B. C. Gamoke, N. J. Mayhall, and K. Raghavachari \\
\textsl{Journal of Chemical Theory and Computation}, 8, 5132-5136, (2012)

\item [16\hspace{4pt}]
\textbf{Many-Overlapping-Body (MOB) Expansion: A Generalized Many Body Expansion for Nondisjoint Monomers in Molecular Fragmentation Calculations of Covalent Molecules} \\
N. J. Mayhall and K. Raghavachari \\
\textsl{Journal of Chemical Theory and Computation}, 8, 2669-2675, (2012)

\item[15\hspace{4pt}] \textbf{Modeling Nonperiodic Adsorption on Periodic Surfaces: A Composite Energy Approach for Low-Coverage Limits}\\
B. C. Gamoke, N. J. Mayhall, and K. Raghavachari \\
\textsl{Journal of Physical Chemistry C}, 116, 12048-12054, (2012)

\item[14\hspace{4pt}] \textbf{Properties of metal oxide clusters in non-traditional oxidation states}\\
J. E. Mann, N. J. Mayhall, and C. C. Jarrold\\
 \textsl{Chemical Physics Letters}, 525-526, 1-12, (2012)

\item[13\hspace{4pt}] \textbf{Molecules-in-Molecules: A Hybrid-Energy Fragmentation Approach for Accurate Calculations on Large Molecules and Materials}\\
N. J. Mayhall and K. Raghavachari\\
\textsl{Journal of Chemical Theory and Computation}, 7, 1336-1343, (2011)

\item[12\hspace{4pt}] \textbf{Molybdenum Oxides vs. Molybdenum Sulfides: Geometric and Electronic Structures of Mo$_3$X$_\text{y}^−$ (X=O, S and y=6, 9) Clusters}\\
N. J. Mayhall, E. L. Becher, A. Chowdhury, K. Raghavachari\\
\textsl{Journal of Physical Chemistry A}, 115, 2291-2296, (2011)

\item[11\hspace{4pt}] \textbf{Charge transfer across ONIOM QM:QM boundaries: The impact of model system preparation}\\
N. J. Mayhall and K. Raghavachari\\
 \textsl{Journal of Chemical Theory and Computation}, 6, 3131-3136 (2010)

\item[10\hspace{4pt}] \textbf{A Proton Hop Paves the Way for Hydroxyl Migration: Theoretical Elucidation of Fluxionality in Transition Metal Oxide Clusters}\\
R. Ramabhadran, N. J. Mayhall, K. Raghavachari\\
\textsl{Journal of Physical Chemistry Letters}, 1, 3066-3071 (2010)

\item[9\hspace{4pt}] \textbf{Multiple solutions to the single-reference CCSD equations for NiH}\\
N. J. Mayhall, K. Raghavachari\\
\textsl{Journal of Chemical Theory and Computation}, 6, 2714 (2010)

\item[8\hspace{4pt}] \textbf{ONIOM-based QM:QM electronic embedding method using L\"owdin atomic charges: Energies and analytic gradients}\\
N. J. Mayhall, K. Raghavachari, H. P. Hratchian\\
\textsl{Journal of Chemical Physics}, 132, 114107 (2010)

\item[7\hspace{4pt}] \textbf{Termination of the W$_2$O$_\text{y}^-$ + H$_2$O/D$_2$O $\rightarrow$W$_2$O$_{\text{y+1}}^-$+H$_2$/D$_2$ sequential oxidation reaction: An exploration of kinetic versus thermodynamic effects}\\
D. W. Rothgeb, E. Hossain, N. J. Mayhall, K. Raghavachari, C. C. Jarrold\\
\textsl{Journal of Chemical Physics}, 131, 144306 (2009)


\item[6\hspace{4pt}] \textbf{Water Reactivity with Tungsten Oxides: H$_2$ Production and Kinetic Traps}\\
N. J. Mayhall, D. W. Rothgeb, E. Hossain, C. C. Jarrold, K. Raghavachari\\
\textsl{Journal of Chemical Physics}, 131, 144302 (2009)


\item[5\hspace{4pt}] \textbf{Electronic structures of MoWO$_\text{y}^-$ and MoWO$_\text{y}$ determined by anion photoelectron spectroscopy and DFT calculations}\\
N. J. Mayhall, D. W. Rothgeb, E. Hossain, K. Raghavachari, C. C. Jarrold\\
\textsl{Journal of Chemical Physics}, 130, 124313 (2009)


\item[4\hspace{4pt}] \textbf{Investigation of G4 Theory for Transition Metal Thermochemistry}\\
N. J. Mayhall, K. Raghavachari, P. C. Redfern, L. A. Curtiss\\
\textsl{Journal of Physical Chemistry A}, 113 5170-5175 (2009)


\item[3\hspace{4pt}] \textbf{Unusual products observed in gas-phase W$_\text{x}$O$_\text{y}^-$ + H$_2$O and D$_2$O reactions}\\
D. W. Rothgeb, E. Hossain, A. T. Kuo, J. L. Troyer, C. C. Jarrold, N. J. Mayhall, K. Raghavachari\\
\textsl{Journal of Chemical Physics}, 130, 124314 (2009)


\item[2\hspace{4pt}] \textbf{Toward accurate 
thermochemical models for transition metals: G3Large basis sets for atoms Sc-Zn}\\
N. J. Mayhall, K. Raghavachari, P. C. Redfern, L. A. Curtiss, V. Rassolov\\
\textsl{Journal of Chemical Physics}, 128, 144122 (2008)


\item[1\hspace{4pt}] \textbf{Two Methanes are Better than One: A Density Functional Theory Study of the Reactions of Mo$_{2}$O$_\text{y}^-$ (y = 2-5) with Methane}\\
N. J. Mayhall, K. Raghavachari\\
\textsl{Journal of Physical Chemistry A}, 111, 8211-8217 (2007)
%\end{revnumerate}
\end{lonelist}

\blankline

\section{\textbf{\color{red}Book Chapters}}
\begin{lonelist}
\item[$\bullet$] \textit{Energy Transfer in Metal Organic Frameworks}\\
	J. Zhu, S. Shaikh, N. J. Mayhall, \& A. Morris \\
	in \textit{Elaboration and Applications of Metal-Organic Frameworks} \\
	Editor: S. Ma \\
	World Scientific Publishers/Imperial College Press (2017)
\end{lonelist}

\blankline

\section{\textbf{\color{red}Journals Refereed}}
\begin{lonelist}
	  \setlength\itemsep{1pt}
\item[] Journal of Chemical Physics 
\item[] Journal of Chemical Theory and Computation
\item[] Journal of Physical Chemistry
\item[] Physical Chemistry Chemical Physics
\item[] Journal of Physical Chemistry Letters
\item[] Molecular Physics
\item[] Chemical Physics Letters 
\end{lonelist}

\blankline
%\pagebreak

%\section{\textbf{Conference Presentations}}
\section{\textbf{\color{red}Oral Presentations}}
\begin{lonelist}

\item[$\bullet$] \textbf{Invited Talk:} \textit{Using quantum chemistry to simulate SMM qubits to (someday) simulate quantum chemistry}\\
Gordon Research Conference, Computational Chemistry, West Dover, VT (2018)

\item[$\bullet$] \textbf{Invited Talk:} \textit{Using single-excitation wavefunctions to compute exciton-binding energies in singlet fission materials}\\
East Tennessee State University, Johnson City, TN  April 13 (2018)

\item[$\bullet$] \textit{Using higher-order singular value decomposition to define weakly coupled and strongly correlated clusters: the n-body Tucker approximation}\\
255th ACS National Meeting,  March 21 (2018)

\item[$\bullet$] \textit{A generalized Ovchinnikov's rule can predict the biexciton boundedness in covalently linked singlet fission chromophores}\\
255th ACS National Meeting,  March 19 (2018)

\item[$\bullet$] \textit{Multiexcitons and strong correlation via single-excitation wavefunctions: applications and future directions}\\
2017 WATOC, Munich, Germany,  August 31 (2017)

\item[$\bullet$] \textbf{Invited Talk:} \textit{Spin flip methods for Spin Hamiltonians}\\
New Frontiers in Electron Correlation, Telluride TSRC, CO,  June 23 (2017)

\item[$\bullet$] \textit{Using single-excitation wavefunctions to compute exciton-binding energies in singlet fission materials}\\
253rd ACS National Meeting, San Francisco, CA, April 4 (2017)

\item[$\bullet$] \textbf{Invited Talk:} \textit{Using simple ab initio methods to construct even simpler Hamiltonians: applying spin-flip methods for strong correlation and excited states}\\
Joint Condensed Matter and Center for Soft Matter and Biological Physics Seminar, Virginia Tech, Nov. 14 (2016)

\item[$\bullet$] \textbf{Invited Talk:} \textit{Using single-excitation wavefunctions to compute exciton-binding energies in singlet fission materials}\\
Department Seminar, James Madison University, Nov. 11 (2016)

\item[$\bullet$] \textbf{Invited Talk:} \textit{Using simple ab initio methods to construct even simpler Hamiltonians: spin-flip methods for strong correlation and excited states}\\
SETCA, Tallahassee, FL (2016)

\item[$\bullet$] \textbf{Invited Talk:} \textit{Ab initio Quantum Chemistry for multiradical molecules: A spin-flip approach}\\
251st ACS National Meeting, San Diego, CA (2016)

\item[$\bullet$] \textbf{Invited Talk:} \textit{Heisenberg spin couplings are difficult but not impossible: Ab initio Quantum Chemistry for multiradical molecule}\\
UC Merced, Merced, CA (2015) 

\item[$\bullet$] \textbf{Invited Talk:} \textit{Toward accurate single-reference descriptions of strongly correlated systems: Spin-flip methods for several coupled electrons}\\
248st ACS National Meeting, San Francisco, CA (2014) 

\item[$\bullet$] \textit{Cost effective modeling of spin-coupled molecules: A 2nd order perturbative treatment of orbital relaxation in complete active space spin-flip CI}\\
246st ACS National Meeting, Indianapolis, IN (2013) 

\item[$\bullet$] \textit{Improving hybrid energy schemes for large molecules: Inclusion of charge-redistribution across regional boundaries}\\ 
241st ACS National Meeting, Anaheim, CA (2011) 

\item[$\bullet$] \textbf{Invited Talk:} \textit{Composite Energy Models in Quantum Chemistry} \\
UC Berkeley, CA (2011)

\item[$\bullet$] \textit{H$_2$ Production and Kinetic Traps: Water Reactivity with Tungsten Oxides}\\ 
65th International Symposium on Molecular Spectroscopy, The Ohio State University, Columbus, OH (2010) 

\item[$\bullet$] \textit{First Principles Determination of the Acetyl Anion Photoelectron Spectrum}\\
Undergraduate Research Conference, Butler University, Indianapolis, IN (2006)\\\\
\end{lonelist}


\section{\textbf{\color{red}Poster Presentations}}
\begin{lonelist}

\item[$\bullet$] \textit{Active space-based spin-flip methods for strongly correlated systems: Method development and phenomenological extensions}\\ 
SciDAC-3 PI Meeting, Washington DC (2014) 

\item[$\bullet$] \textit{A perturbative approximation to RAS-$n$SF for excited states and strong correlations}\\ 
SciDAC-3 PI Meeting, Rockville, MD (2013) 

\item[$\bullet$] \textit{Is Coupled Cluster a Black Box?}\\
42st Midwest Theoretical Chemistry Conference, Purdue University, West Lafayette, IN (2010)

\item[$\bullet$] \textit{H$_2$ Production and Kinetic Traps: Water Reactivity with Tungsten Oxides}\\
41st Midwest Theoretical Chemistry Conference, University of Southern Illinois, Carbondale, Ill (2009)

\item[$\bullet$] \textit{Transition Metal Thermochemistry with Composite Methods: Application and Assessment of Current Methods}\\
40th Midwest Theoretical Chemistry Conference, University of Michigan, Ann Arbor, MI (2008)

\item[$\bullet$] \textit{Two Methanes are Better than One: A Density Functional Theory Study of the Reactions of Mo$_{2}$O$_y^-$ (y = 2-5) with Methane}\\
39th Midwest Theoretical Chemistry Conference, Indiana University, Bloomington, IN  (2007) 
\end{lonelist}

\blankline

%\section{Programming Languages} 
%
%Perl, Fortran77, GNU Octave, and UNIX shell scripting
%\textsl{Chemistry Related Software}: Gaussian (\href{http://www.gaussian.com}{www.gaussian.com}), Gamess, MolPro, and NWChem and various molecular visualization packages, GaussView, VMD, Chemcraft
%
%\blankline
%
%\textsl{Programming}: Perl, Fortran77, programming in Gaussian, GNU Octave, and UNIX shell scripting
%
%\blankline
%
%%\textsl{Applications}: \LaTeX{}, Microsoft Office, MediaWiki, DekiWiki, VirtualBox and other common productivity packages for Windows and Linux platforms
%%
%%\blankline
%
%\textsl{Operating Systems}: Microsoft Windows XP/2000/Vista, Linux 


\end{document}

%%%%%%%%%%%%%%%%%%%%%%%%%% End CV Document %%%%%%%%%%%%%%%%%%%%%%%%%%%%%
